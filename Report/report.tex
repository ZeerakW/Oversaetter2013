\documentclass[10pt]{article}
\usepackage[margin=3cm]{geometry}
\usepackage[utf8]{inputenc}
\usepackage[T1]{fontenc}
\usepackage{array}
\usepackage{amsmath}
\usepackage{xcolor}
\usepackage{hyperref}
\usepackage{graphicx}
\usepackage{epstopdf}

\title{A Compiler for the \textbf{Paladim} Language}
\author{
    Klaus Møllnitz\\
    nhg665
  \and
    Konstantin Slavin-Borovskij\\
    xxx123 % Fill in ku numberplate
  \and
    Zeerak Waseem\\
    xxx123 % Fill in ku numberplate
}
\begin{document}
\maketitle

\section{Introduction}
Content goes here.

\section{Background}

\section{Analysis}
\subsection{Parser}
\subsubsection{Identifiyng tokens}
In our abstract parser definition most of our tokens use the type $<int*int>$. This type is the position of the token, which is constructed by line and character. If nothing is stated in the tables, this is the type used.

We have ifdentified the list of terminals and given everyone of them keywords, for use later in the abstract parser definition.

All the keywords of the Paladim language is listed in table \ref{tab:keywords}, these only use the positional data, meaning they all have token type $<int*int>$, since we have chosen to denote the position by linenumber and character in the line.

\begin{table}[h!]
\centering
\begin{tabular}{c|c}
Terminal & Keyword \\
\hline
\textbf{program} & PROGRAM \\
\textbf{function} & FUNCTION \\
\textbf{procedure} & PROCEDURE \\
\textbf{var} & VAR \\
\textbf{begin} & BEGIN \\
\textbf{end} & END \\
\textbf{if} & IF \\
\textbf{then} & THEN \\
\textbf{else} & ELSE \\
\textbf{while} & WHILE \\
\textbf{do} & DO \\
\textbf{return} & RETURN \\
\end{tabular}
\caption{\label{tab:keywords}Keywords}
\end{table}

We also need keywords for the specific types, since we should be able to write the following as noted in (\ref{eq:varname_type}). The keyword is written in Table \ref{tab:type_keywords}.
\begin{align}
\label{eq:varname_type}
\text{VariableName} \textbf{ : } \text{Type}.
\end{align}

\begin{table}[h!]
\centering
\begin{tabular}{c|c}
Terminal & Keyword \\
\hline
\textbf{int} & INT \\
\textbf{char} & CHAR \\
\textbf{bool} & BOOL \\
\textbf{array of} & ARRAYOF \\
\textbf{of} & OF \\
\end{tabular}
\caption{\label{tab:type_keywords}Type keywords}
\end{table}

Next terminals we need to consider is symbols. These are very simple and shown in table \ref{tab:symbols}.

\begin{table}[h!]
\centering
\begin{tabular}{c|c}
Terminal & Keyword \\
\hline
\textbf{;} & SEMICOLON \\
\textbf{:} & COLON \\
\textbf{,} & COMMA \\
\textbf{:=} & ASSIGN \\
\textbf{EOF} & EOF \\
\end{tabular}
\caption{\label{tab:symbols}Symbols}
\end{table}

All arithmetic operators is found in table \ref{tab:arithmetics}.

\begin{table}[h!]
\centering
\begin{tabular}{c|c}
Terminal & Keyword \\
\hline
\text{+} & PLUS \\
\text{-} & MINUS \\
\text{*} & TIMES \\
\text{/} & DIVIDE \\
\text{=} & EQUAL \\
\text{<} & LESS \\
\textbf{AND} & AND \\
\textbf{OR} & OR \\
\textbf{NOT} & NOT \\
\end{tabular}
\caption{\label{tab:arithmetics}Arithmetics}
\end{table}

Parantheses is covered in table \ref{tab:parentheses}.

\begin{table}[h!]
\centering
\begin{tabular}{c|c}
Terminal & Keyword \\
\hline
\text{(} & LPAR \\
\text{)} & RPAR \\
\text{\{} & LCBR \\
\text{\}} & RCBR \\
\text{[} & LBRA \\
\text{]} & RBRA \\
\end{tabular}
\caption{\label{tab:parentheses}Parentheses}
\end{table}

Finally we have all of the literals, which are supposed to contain both data and a position. As seen in table \ref{tab:literals}, they all make use of some sort of variable to store their content. We also cover the Identifier here.

\begin{table}[!h]
\centering
\begin{tabular}{c|cc}
Terminal & Keyword & Type\\
\hline
\textbf{ID} & ID & <string*(int*int)>\\
\textbf{NUMLIT} & NUMLIT & <int*(int*int)>\\
\textbf{LOGICLIT} & LOGLIT & <bool*(int*int)> \\
\textbf{CHARLIT} & CHALIT & <char*(int*int)> \\
\textbf{STRINGLIT} & STRLIT & <string*(int*int)> \\
\end{tabular}
\caption{\label{tab:literals}Literals}
\end{table}

\subsubsection{Precedence}
Next tasks is to denote the precedence of the operators.
Write here \ldots

\subsubsection{Non-terminals}
Write here \ldots

\subsubsection{Productions}
Write here \ldots

\subsection{Integer multiplication/division}
\subsection{Boolean operators}
\subsection{Type inteference}
\subsection{Type checking}
\subsection{Array indexing}
\subsection{Call-by-value-result semantics}

\section{Testing}

\section{Conclusion}

\section{Bibliography}

%\section{Appendix A: }

\end{document}
