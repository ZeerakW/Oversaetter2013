\documentclass{article}
\usepackage{graphicx}
\usepackage[dot, autosize, outputdir="graphfiles/"]{dot2texi}
\usepackage{tikz}
\usepackage{amsmath, mathtools}
\usetikzlibrary{shapes}
\title{W3-Resubmission}
\author{Zeerak Waseem - csp265}
\date{14/12/2013}
\begin{document}
\maketitle
\newpage

\section{Exercise 1}

\subsection{a}
When f(a, b, c) returns, it returns \(7 + 3 + 12 = 22\). Thus when call-by-value is used, \(r = 22\) \(x\) and \(y\) remain unchanged in the calling function. The reason is that call-by-value creates a copy of the variables to be used in the called function, thus leaving the variables unchanged in the calling function.\\
Prints: \(r=22, x = 5, y = 2\). Obviously it doesn't print the variable names, just their values.
\subsection{b}
When call-by-reference is used, the variables in the calling function are modified, thus \(x\) to be first \(7\) then subsequently changing \(x\) again to be \(14\). \(y\) is changed to \(3\).\\
Prints: \(r = 33, x = 14, y = 5\).
\subsection{c}
When call-by-value-result is used, the variables are only changed when the called function returns, leaving the values in the calling function unchanged (as such, \(x\) would only change once rather than twice, as with call-by-reference. Thus \(x\) and \(y\) would be modified to \(12\) and \(3\) respectively.\\
Prints: \(r = 22, x = 12, y = 3\).
\section{Exercise 2}
\subsection{a}
It would print 5 twice as g() has no access to the local \(x's\), but only to the global.
\subsection{b}
It would print 4 and 9, as g() does have access to the local variables.
\subsection{c}
By editing in the \texttt{TpInterpret.sml} in the \texttt{callFun} function. By removing the symbol table, the variables would be referenced by the variables in the calling function, thus giving access to the local values.  

\section{Exercise 3}
\subsection{a}
The type checker starts by matching on the equality, it then checks the left side of the expression, where an \texttt{int} is passed to the \texttt{chr()} function, which outputs a \texttt{char}, by means of the \texttt{read()} function. This gives us the knowledge that the right hand side is expected to be a \texttt{char}. 
\subsection{b}
\end{document}
